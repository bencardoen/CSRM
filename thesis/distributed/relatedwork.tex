
The reference work for the field of distributed metaheuristcs \cite{parallelmetaheuristics} provides a broad overview of the challenges and advantages of parallel metaheuristics. SR can be implemented using a parallel metaheuristic such as GP.
An interesting parallel GP SR implementation \citep{DGPSR} introduces a random islands model where processes are allowed to ignore messages, contrary to our approach. The authors argue that this promotes niching where 'contamination' of locally (per process) fit individuals could otherwise introduce premature convergence. The clear advantage of such a system is speedup, since no process ever waits on other processes. Another difference is the message exchange protocol. Whereas our tool exchanges messages after each phase, their tool uses a probability to decide per process if messages are sent or received interleaved with the generations. Such a setup allows for a heterogeneous set of processes. A different approach is shown in \citep{DFGPSR} where a master slave topology is used in combination with a load balancing algorithm in order to resolve the imbalance between the different slaves executing uneven workloads. The slaves do not form separate processes, they are assigned a subset of the population and execute only the fitness function. The selection and evolution steps are performed by the master process. This a a fine grained approach, and while it offers a speedup in comparison with a sequential GP SR implementation it does not increase the coverage of the search space. In Distributed Genetic Progamming (DGP) \cite{DGP} a ring and torus topology are used. The two way torus topology is similar to our grid topology. The study finds that sharing of messages is essential to improve convergence but that the communication pattern is largely defined by the problem domain. It concludes that diffusion is a more powerful technique compared to partitioning. In partitioning no communication between subgroups is possible, which can protect against premature convergence.
