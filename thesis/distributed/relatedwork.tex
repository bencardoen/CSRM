%%!TEX root = document.tex

The reference work of E. Alba \cite{parallelmetaheuristics} provides a broad overview of the challenges and advantages of parallel metaheuristics. 
In a random island model for a parallel GP-SR implementation \citep{DGPSR}, processes are allowed to ignore messages. The authors argue that this promotes niching, where `contamination' of locally (per process) fit expressions could introduce premature convergence. The advantage of such a system is speed-up, since no process has to wait on other processes. Another difference with our approach is the message exchange protocol. 
This approach interleaves message exchange with computation during a phase, which allows for a heterogeneous set of processes. 

Another approach is a master slave topology in combination with a load balancing algorithm between the slaves \citep{DFGPSR}. The slaves are not separate algorithms: they are assigned a subset of the population and only compute the fitness function. Selection and evolution are performed by the master process. This fine grained approach offers a speed-up compared to a sequential GP-SR, but it does not increase the coverage of the search space. 

Ting and torus topologies are also used in the literature \cite{DGP}. A two-way torus topology is similar to our grid topology. The study of Nigwa et al. \cite{DGP} states that sharing of messages is essential to improve convergence but the communication pattern is largely defined by the problem domain. They conclude that diffusion is more powerful compared to partitioning.